\chapter{Numeri Complessi}\index{Numeri! Complessi}
\thispagestyle{fancy}

		Per poter risolvere certe equazioni come ad esempio $x^2+1=0$ \`{e} necessario un insieme di numeri in pi\`{u}, perch\'{e} tale tipo di equazioni non ha soluzione nel campo dei reali.
		
		L'insieme dei numeri complessi \`{e} indicato con $\C$, mentre i numeri sono indicati con la lettera $i$ detta \emph{unit\`{a} immaginaria}, avente la propriet\`{a} che $i^2 = -1$.
		Cos\`{i} facendo anche le equazioni di $2^o$ grado del tipo $x^2+2px+q=0$ avranno soluzione e se $p^2 < q$ allora la soluzione sar\`{a} $$x = -p \pm i\sqrt{q-p^2}$$
		
		Quindi, i numeri complessi, si posso scrivere nella forma $a+ib$ dove \emph{a} e \emph{b} sono numeri reali ma vengono detti rispettivamente \emph{parte reale} e \emph{parte immaginaria}.
		
		\section{Operazioni con i numeri complessi}
				\subsection{Somma e Prodotto}
				Definendo le operazioni di somma e prodotto tra numeri complessi, c'\`{e} da tenere a mente sempre che $i^2 = -1$, in questo modo possiamo vedere che
				\begin{itemize}
					\item $(a+ib)+(c+id) = (a+c) + i(c+d)$
					\item $(a+ib)(c+id) = (ac-bd)+i(ad+bc)$
				\end{itemize}
			
				\subsection{Modulo}
					$$\alpha = a+ib \in \C \Rightarrow |\alpha| = \sqrt{a^2+b^2}$$
					
					Anche i numeri complessi possono essere messi in relazioni con la retta in corrispondenza biunivoca, avendo l'asse $x$ come \emph{retta reale} e l'asse $y$ come \emph{asse immaginaria}. Creando un grafico, utilizzando questo metodo, \`{e} facile notare che il modulo di un numero complesso corrisponde al calcolare la distanza del punto definito da \emph{a} e \emph{b} rispetto all'origine.
					
					\begin{figure}[ht]
						\centering
						\includegraphics[width=.5\textwidth]{src/Images/GraficoComplex.png}
						\caption{Distanza del punto $\alpha$ dall'origine}
					\end{figure}
				
				\subsection{Rappresentazione Trigonometrica}
					Tale corrispondenza tra numeri complessi e punti del piano e la visione grafica, aiuta a capire che il numero $\alpha$ \`{e} possibile descriverlo anche attraverso la sua distanza dall'origine e l'angolo che lo divide dall'asse dei reali, arrivando ad avere un'equazione del tipo
					$$\alpha = a+ib = d(\cos(\theta) + i\sin(\theta)) = de^{i\theta}$$					
					dove \emph{d} \`{e} la distanza tra il punto e l'origine degli assi, $\theta$ l'angolo che si forma tra l'asse $x$ e il punto.\\
					Nella formula $de^{i\theta}$ chiaramente la parte esponenziale sar\'{a} congruente a $\cos(\theta)+i\sin(\theta)$, e le due formule sono totalmente interscambiabili.
				
				\subsection{Radici}
					Una radice \emph{n}-esima di $\omega$ \`{e} un numero complesso $z$ tale che $z^n = \omega$.\\
					Inoltre\\
					$z = de^{i\theta} \; \land \; \omega = De^{i\delta} \\ \Rightarrow z^{n} = d^{n}e^{in\theta} = De^{i\delta} \\ \Rightarrow d^{n} = D \; \land \; n\theta = \delta + 2k\pi \\ \Rightarrow r = \sqrt[n]{D}$
		