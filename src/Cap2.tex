\chapter[Numeri Interi e Razionali]{Numeri Interi e\\ Razionali}
\thispagestyle{fancy}

	\section{Interi Relativi}\index{Numeri! Relativi}
		I numeri interi relativi $\Z$ sono l'insieme dei numeri naturali $\N$ e i loro opposti, cio\`{e} i negativi. Tali numeri possono essere scritti anche come frazioni, il cui denominatore pu\`{o} assumere valore 1 oppure un multiplo del numeratore.
		$$I \subseteq \Z = \{-1, \frac{3}{1}, \frac{8}{2}\}$$

	\section{Razionali}\index{Numeri! Razionali}
			Se scrivendo un numero fratto ci si accorge che non \`{e} possibile ridurre tale frazione ad un intero, allora siamo incappati in un nuovo insieme di numeri detto \emph{insieme dei numeri razionali}. Per convenzione, tale numero fratto, si descrive come una \textbf{frazione ridotta ai minimi termini}, cio\`{e} dove numeratore e denominatore sono \textbf{coprimi}, il che significa primi tra loro.
		Esempio:
		$$\frac{6}{4} = \frac{3}{2}$$
		dove $\frac{3}{2}$ \`{e} la frazione ridotta ai minimi termini. In oltre \`{e} bene notare che
		\emph{due frazioni $\frac{a}{b}$ e $\frac{c}{d}$ sono congruenti se e solo se $a \cdot d = b \cdot c$} 
